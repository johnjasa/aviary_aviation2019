% this file should contain a summary of the propulsion system analysis tools and models including propeller, electric and cycle

% Summary paragraph
The trajectory is modeled as two two distinct forms of flight.
The trajectory begins and ends with vertical ascent/descent phases which propagate the change in altitude and fuel mass subject to a commanded climb rate.
The bulk of the trajectory is performed under a forward flight mode which is modeled using a steady-flight assumption, described below.
Currently this work ignores the transition phase between vertical and forward flight modes.

\subsubsection{Vertical Flight}

In vertical flight, the trajectory of the aircraft is controlled by specifying the climb rate at any given time.
In this analysis, the climb rate is assumed to be constant during the takeoff and landing phases.
The dynamics during vertical flight are governed by momentum theory\cite{leishman2006principles}.
The role of the vertical flight dynamics system is to determine the rotor thrust and rotor power required to achieve the prescribed rate of climb.

The thrust required to attain a given climb rate is

\begin{align}
    T &= 2 \rho A_{rotor} (\dot{h} + v_h) v_h \label{eq:climb_thrust}
\end{align}

where the induced velocity in hover ($v_h$) is computed as

\begin{align}
    v_h &= \sqrt{\frac{T_{rotor}}{2 \rho A_{rotor}}} \label{eq:hover_induced_velocity}
\end{align}

This thrust is then passed to the propulsion model along with the rate of climb to compute the necessary shaft input power.

\subsubsection{Forward Flight}

Forward flight is modeled using a steady flight assumption originally posed by Bryson et al. whereby the flight path angle is assumed to be nearly zero\cite{bryson1969energystate}.

\begin{align}
    T \alpha + L(\alpha) = W \label{eq:forward_flight_balance}
\end{align}

A Newton solver is used to find the value of $\alpha$ which balances \eqref{eq:forward_flight_balance}.
The total thrust ($T$) of the aircraft in forward-flight is imposed as control variable.
The aircraft lift is provided by the aerodynamics model.
Finally, the vehicle weight ($W$) is computed by summing of the masses of the constituent subsystems.
